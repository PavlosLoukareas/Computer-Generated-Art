\documentclass[a4paper, 12pt]{article}
\usepackage[greek]{babel}
\usepackage[utf8x]{inputenc}
\usepackage{parskip}

\begin{document}
\textbf{Τριγωνομετρικές Συναρτήσεις}\\

\textbf{Aντίθετα Ορίσματα} \\
\noindent
\begin{tabular}{lr}
  $\sin(-x)=-\sin(x)$ & $\cos(x)=\cos(-x)$ \\
  $\tan(-x)=-\tan(x)$ & $\cot(-x)=-\cot(x)$ \\
  $\sec(-x)=\sec(x)$  & $\csc(-x)=-\csc(x)$ \\
\end{tabular}

\textbf{Άθροισμα \& Διαφορά Γωνιών} \\
\noindent
  $\sin(x \pm y)=\sin(x)\cos(y)\pm\cos(x)\sin(y)$ \\
  $\cos(x \pm y)=\cos(x)\cos(y)\mp\sin(x)\sin(y)$ \\
  $\tan(x \pm y)=\frac{\tan(x)\pm\tan(y)}{1\mp\tan(x)\tan(y)}$ \\
  $\cot(x \pm y)=\frac{\cot(x)\cot(y)\mp1}{\cot(x)\pm\cot(y)}$ \\

\textbf{Διπλή Γωνία} \\
\noindent
  $\sin(2x)=2\sin(x)\cos(x)$ \\
  $\cos(2x)=\cos^2(x)-\sin^2(x)=1-2\sin^2(x)=2\cos^2(x)-1$ \\
  $\tan(2x)=\frac{2\tan(x)}{1-\tan^2(x)}$ \\

\textbf{Μισή Γωνία} \\
\noindent
  $\sin(\frac{x}{2})=\pm\sqrt{\frac{1-\cos(x)}{2}}$ \\
  $\cos(\frac{x}{2})=\pm\sqrt{\frac{1+\cos(x)}{2}}$ \\
  $\tan(\frac{x}{2})=\pm\sqrt{\frac{1-\cos(x)}{1+\cos(x)}}=\frac{\sin(x)}{1+\cos(x)}=
  \frac{1-\cos(x)}{\sin(x)}=\csc(x)-\cot(x)$ \\

\textbf{Άθροισμα, Διαφορά \& Γινόμενο} \\
\noindent
  $\sin(x)+\sin(y)=2\sin(\frac{x+y}{2})\cos(\frac{x-y}{2})$ \\ 
  $\sin(x)-\sin(y)=2\cos(\frac{x+y}{2})\sin(\frac{x-y}{2})$ \\ 
  $\cos(x)+\cos(y)=2\cos(\frac{x+y}{2})\cos(\frac{x-y}{2})$ \\ 
  $\cos(x)-\cos(y)=2\sin(\frac{x+y}{2})\sin(\frac{x-y}{2})$ \\ 
  $\sin(x)\sin(y)=\frac{1}{2}[\cos(x-y)-\cos(x+y)]$ \\
  $\cos(x)\cos(y)=\frac{1}{2}[\cos(x-y)+\cos(x+y)]$ \\
  $\sin(x)\cos(y)=\frac{1}{2}[\sin(x-y)+\sin(x+y)]$ \\
  
Οι ίδιες ιδιότητες ισχύουν και για τις υπερβολικές τριγωνομετρικές συναρτήσεις,
με τη διαφορά ότι $\sinh(x)=\frac{e^{x}-e^{-x}}{2}$ \&
$\cosh(x)=\frac{e^{x}+e^{-x}}{2}$
\end{document}
